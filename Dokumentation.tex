% AlgoProjekt Dokumentation.
% Geschrieben in Gummi.
\documentclass[11pt]{article}
\usepackage[utf8]{inputenc}
\usepackage[ngerman]{babel} 
\usepackage{lmodern}            
\title{\textbf{AlgoProjekt}}
\author{Maria Markstäadter\\
                Noel Kuntze\\
                Ruben Anders\\}
\date{\today}
\begin{document}

\maketitle

\newpage

\tableofcontents

\newpage

  \section{Unser Programm}
Dieses Dokument ist ein Teil des Projekts der Autoren im Rahmen ihres Studiums an der Hochschule Offenburg.
  \subsection{Programmkurzbeschreibung}
  sdsds
  \subsection{Programmkenndaten}
  \subsubsection{Programmidentifizierung}
  \paragraph{Programmname}
  AlgoProjekt
  \paragraph{Systemzuordnung des Programms}
  Kryptologie
  \paragraph{Programmversion}
  Version 1.0
  \paragraph{Freigabedatum}
  \date{26.11.2013}
  \section{Programmfunktion}
  dsdsads
  \subsection{Aufgabenstellung}
  Wir wollen ein einfaches, passwortbasiertes Zugangssystem angreifen. Das Zugangssystem
funktioniert wie folgt:
\begin{itemize} 
\item Der Nutzer, der sich authentifizieren will, gibt an einem gesicherten Terminal sein
Passwort ein. Der String wird mit Hilfe der kryptographischen Hashfunktion SHA-1
gehasht. Verwendet werden allerdings nur die letzten 4 Byte.
\item Das Terminal sendet diesen sogenannten Message Digest zum Authentifikationsser-
ver. Dieser vergleicht den erhaltenen Message Digest mit dem Wert, den er für die-
sen Nutzer gespeichert hat. Sind die beiden identisch, wird der Zugang freigegeben,
ansonsten nicht.
\end{itemize}
  \subsection{Theoretische Grundlagen}
  \subsection{Funktionshirachie}
  \subsection{Methoden/Algorithmen}
  
  
  \section{Programmaufbau}
  \subsection{Programmbausteine}
  \subsection{Programmstruktur}
  
  
  \section{Programmablauf}
  \subsection{Programmablaufbeschreibung} 
  \subsection{Funktionshirachie}
  \subsection{Methoden/Algorithmen}
  
   \section{Programmtest}
  \subsection{Testziele}
  \subsection{Testverfahren}
  \subsection{Testfälle/Testresultate}
  \dots
   
 
\newpage

\end{document}
