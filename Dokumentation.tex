% AlgoProjekt Dokumentation.
% Geschrieben in Gummi.
\documentclass[11pt]{article}
\usepackage[utf8]{inputenc}
\usepackage[ngerman]{babel} 
\usepackage{lmodern}            
\title{\textbf{AlgoProjekt}}
\author{Maria Markstädter\\
                Noel Kuntze\\
                Ruben Anders\\}
\date{\today}
\begin{document} 

\maketitle
 
\tableofcontents
  \section{Unser Programm}
Dieses Dokument ist ein Teil des Projekts der Autoren im Rahmen ihres Studiums an der Hochschule Offenburg.
  \subsection{Programmkurzbeschreibung}
  Szenario:
Über ein Terminal wird ein Passwort eingegeben und  mittels der Hashfunktion SHA-1 verschlüsselt. Die letzten 4 Bytes (sog. Message Digest) werden über eine ungesicherte Leitung übertragen und am Authentifikationsserver mit dem Wert der für den Nutzer gespeichert wurde, verglichen. \newline 
Ziel ist es, ein gesnifftes Message Digest mit einer zuvor berechneten Hashtabelle zu vergleichen und so das Passwort herauszufinden.\vspace{2px}   \newline 
{\itshape{Dies findet in 2 Phasen statt:}} \vspace{2px}   \newline
Precomputation-Phase: Es werden Passwort und Message Digest berechnet und in einer Hashtable gespeichert.\vspace{2px} \newline 
Online-Phase: Der gesniffte Message Digest wird mit der zuvor berechneten Hashtable verglichen und so das dazugehörige Passwort herausgelesen. 
Es gilt mehrere Module zu Programmieren, die zum einen eine Hashtable erzeugen in der Passwort und Message Digest berechnet werden sowie in einer Hashtable einen zuvor gesnifften Message Digest nachschlagen und das dazugehörige Passwort ausgeben. Dies wird durch Interface unterstützt die es dem Benutzer ermöglicht einzelne Phasen anzuwählen und erstellte Hashtables zu speichern oder zu laden.

  \subsection{Programmkenndaten}
  dsadsad
  
  \section{Programmfunktion}
  dsdsads
  \subsection{Aufgabenstellung}
  Wir wollen ein einfaches, passwortbasiertes Zugangssystem angreifen. Das Zugangssystem
funktioniert wie folgt:
\begin{itemize} 
\item Der Nutzer, der sich authentifizieren will, gibt an einem gesicherten Terminal sein
Passwort ein. Der String wird mit Hilfe der kryptographischen Hashfunktion SHA-1
gehasht. Verwendet werden allerdings nur die letzten 4 Byte.
\item Das Terminal sendet diesen sogenannten Message Digest zum Authentifikationsser-
ver. Dieser vergleicht den erhaltenen Message Digest mit dem Wert, den er für die-
sen Nutzer gespeichert hat. Sind die beiden identisch, wird der Zugang freigegeben,
ansonsten nicht.
\end{itemize}
  \subsection{Theoretische Grundlagen}
  \subsection{Funktionshierachie}
  \subsection{Methoden/Algorithmen}
Unter Serialisierung versteht man bei der Java Programmierung das umwandeln eines Objekts in einen Bytestrom der auf einen Datenträger gespeichert werden kann und somit portabel ist. Das Objekt liegt nach der Serialisierung im Arbeitsspeicher und z.B. auf einem Datenträger.
In unserem Fall ermöglicht die Serialisierung des Objekts <table> die zuvor erstellte Hashtable in einen vom User definierten Pfad und Dateinamen zu speichern. Es wird automatisch die Endlung <.ser> eingefügt, da von Sun empfohlen wird diese bei serialisierten Java Objekten zu verwenden.
 
  
  
  \section{Programmaufbau}
  \subsection{Programmbausteine}
  \subsection{Programmstruktur}
  
  
  \section{Programmablauf}
  \subsection{Programmablaufbeschreibung} 
  \subsection{Funktionshirachie}
  \subsection{Methoden/Algorithmen}
  
   \section{Programmtest}
  \subsection{Testziele}
  \subsection{Testverfahren}
  \subsection{Testfälle/Testresultate}
   
   
 
\newpage

\end{document}
